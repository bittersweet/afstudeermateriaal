\chapter{Conclusie}

Door een project uit te voeren bij een bedrijf en een adviesrapport op te stellen heb ik mijn afstudeerproject op een iets andere manier aangepakt dan mijn medestudenten. Het ligt niet helemaal in de lijn van een ``echte'' scriptie, maar ik zie dat toch als een voordeel. Het adviesrapport kan ik straks namelijk aan diverse mensen tonen, die daardoor hun eigen situatie kunnen bekijken en vervolgens -- indien nodig -- misschien ook een aantal adviezen kunnen implementeren. Bij een scriptie heb ik toch meer het idee dat het vrij snel op een plank terecht komt, waarna niemand er ooit meer in kijkt.

Naast de mogelijkheid dat er misschien mensen hun voordeel mee kunnen doen, ben ik ook tevreden over het feit dat ik voor Fullmoon echt iets heb kunnen betekenen. Men liep al een tijdje rond met de gedachte om hun manier van werken te verbeteren, maar omdat het zo druk was had men daar geen tijd voor. Bij Fullmoon waren ze dus blij dat iemand van buitenaf hun hierbij kon assisteren. Tijdens mijn afstudeerstage hebben ze mij geweldig geholpen met het delen van hun kennis en ervaring en ik heb het gevoel dat ik dat nu terug heb kunnen betalen.

Ontwikkeling is vaak niet echt tastbaar maar wanneer ik nu terugkijk op het afgelopen jaar, dan wordt het voor mijzelf toch duidelijk dat deze zeker heeft plaatsgevonden. Dankzij de stage en het project heb ik nu het gevoel gekregen dat ik een volwaardige werknemer ben, die een aanwinst kan zijn voor elk bedrijf. Ik voel dat ik er klaar voor ben om te gaan werken en alles in praktijk te brengen wat ik de afgelopen vier jaar via mijn studie aan de Hogeschool van Amsterdam, opleiding Interactieve Media en door de dagelijkse praktijk heb geleerd.
