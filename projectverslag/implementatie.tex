\chapter{Implementatie}

In het adviesrapport zijn een vijftal adviezen opgesteld, deze komen punt voor punt aan bod zodat na te gaan is wat er precies gebeurd is bij Fullmoon.

\section{SCM Systeem}

\subsection{Keuze}

Voor Fullmoon is de keuze van het {\sc scm} systeem gevallen op Subversion. Één van de uitgangspunten van het project was dat de nieuwe situatie te gebruiken moest zijn voor alle werknemers. Het moest dus een niet te ingrijpende verandering vormen. Omdat het een diverse groep is van designers, programmeurs en projectmanagers, waarbij niet iedereen zich even op het gemak voelt met de commandline, is er gekozen voor Subversion. Het grote aanbod aan (grafische) tools gaf hierbij de doorslag.

\subsection{Uitwerking}

Bij Fullmoon zijn presentaties en uitleg gegeven over het werken met Subversion door middel van het programma SmartSVN. Maar de werknemers zijn in principe vrij in de keuze voor een programma om met Subversion te werken.

Er is een server in gebruik genomen om de Subversion repository te huisvesten. Op dezelfde server draait ook een webserver waarop alle projecten terug te zien zijn. Door middel van \emph{post-commit hooks} worden deze projecten automatisch geupdate na elke commit. Zo is er een centrale plek waar iedereen de laatste versie van het project kan bekijken om deze eventueel te testen. Ook is dit handig om snel een website te kunnen bekijken zonder alles binnen te hoeven halen.

\subsection{Koppeling}

Naast het feit dat elke werknemer nu werkt met Subversion komt het ook nog in andere gebieden terug. Zo wordt er gebruik gemaakt van Subversion om code naar verschillende omgevingen te deployen en is er een centrale website ingericht om de activiteit van projecten te kunnen bekijken. Dit is na te lezen in respectievelijk sectie 2.2 en 2.4.

\section{Ontwikkelstraat}

In het begin maakte Fullmoon gebruik van twee verschillende omgevingen, één interne webserver waar op ontwikkeld werd en een externe server waar alle sites op draaiden. Dat aantal is vergroot naar vier door een acceptatie- en testomgeving toe te voegen. De huidige situatie is dus in de vorm van een OTAP-straat\footnote{ontwikkel, test, acceptatie en productie}. Er wordt niet voor elk project gebruik gemaakt van de acceptatie omgeving maar in principe gaat nieuwe code alle drie de omgevingen door voor het online komt te staan.

De ontwikkelomgeving is verplaats van de centrale server naar de workstations van de medewerkers zelf. Door middel van Subversion werkt iedereen nu op zijn eigen computer. Zo kan iedereen individueel werken aan projecten zonder dat ze elkaar hierbij in de weg zitten. Door middel van Subversion zijn de wijzigingen van anderen uiteraard nog wel binnen te halen.

Zoals vermeld in 2.1.2 wordt de testomgeving automatisch geupdate in navolging van een commit. Hier kan een project intern getest worden door alle werknemers. Omdat niet iedereen binnen Fullmoon direct aan projecten werkt, kunnen deze personen van deze omgeving gebruik maken om projecten te kunnen bekijken zonder dat ze deze zelf op hun computer moeten binnenhalen.

De acceptatieomgeving is ook een interne webserver die afgeschermd is voor de buitenwereld door middel van een gebruikersnaam en wachtwoord. Klanten kunnen hierop inloggen om bijvoorbeeld wijzigingen aan een site te zien voordat deze online komt. Door middel van Subversion wordt een bepaalde versie van het project een \emph{tag} meegegeven, zodat de acceptatieomgeving de website kan tonen op basis van deze tag. Zo kan er verder gewerkt worden aan het project terwijl er controle is over welke versie de klant te zien krijgt om wijzigingen goed te keuren.

\section{Backup}

Bij Fullmoon is de backupsituatie al vrij goed geregeld. Intern worden van alle servers dagelijks een backup gemaakt die bewaard wordt op een externe hardeschijf. Online verzorgd de hostingprovider elke nacht een backup, welke een maand lang worden bewaard op een externe locatie. Deze blijven ook op de server zelf beschikbaar, dit is handig om bijvoorbeeld een database te herstellen mocht er iets fout mee gegaan zijn.

Voor dit onderdeel heb ik weinig kunnen betekenen voor Fullmoon, de situatie was al ruim voldoende ingericht. Alleen is een rampsituatie als degene die ik in het adviesrapport aanhaalde nog steeds mogelijk, bij een brand gaat alle data verloren. Daarom probeer ik nog te adviseren deze backups ook offsite te bewaren, maar daarover zijn we nog in overleg.

\section{Ticketsysteem}

Na vele systemen uitgeprobeerd te hebben is de keuze gevallen op Redmine. Deze sloot nauw aan bij de wensen van Fullmoon. Zo moest er een centrale plek komen waar de status van projecten te bekijken was, zowel issues als wijzigingen in de code.

Fullmoon gebruikt het om het moment alleen intern, ze leveren sinds kort wel support voor de klant met behulp van een ticketsysteem maar dat staat los van Redmine. Mail naar een support adres wordt opgevangen door dit systeem waarna klanten verdere correspondentie en updates ook via de website kunnen raadplegen. In de toekomst zal nog geëvalueerd worden of die scheiding gunstig is want Redmine kan ook voor deze doeleinden worden gebruikt.

\section{Automatisch deployen}

Om code naar verschillende omgevingen toe te sturen wordt gebruik gemaakt van Capistrano in combinatie met de in het advies genoemde multistage plugin. Zo kan met één enkel commando code vanuit de Subversion repository naar de gewenste omgeving gestuurd worden.

Voordat hier gebruik gemaakt van werd ging het updaten van websites niet altijd even goed. Als meerdere mensen lang aan iets gewerkt hadden was het niet meer goed duidelijk wat nou precies veranderd was en wat er nodig was om up te loaden. Zo kwam het voor dat sommige functionaliteit niet goed werkte omdat er een bestand mistte, of dat het verkeerde bestand was geupload. Nu worden sites zonder gevaar en moeite geupdate, en is het een snel klusje geworden waar niet meer naar omgekeken hoeft te worden.

In samenwerking met Subversion bied Capistrano ook de mogelijkheid om snel na te gaan welke versie van een project online staat. Voordat gebruik werd gemaakt van Subversion was het giswerk wat er allemaal online stond, nu kan je direct zien vanaf welke revisie een website draait.
