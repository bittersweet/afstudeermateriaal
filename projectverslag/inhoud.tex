\chapter{Inleiding}

Voor de afstudeerfase van de opleiding Interactieve Media aan de Hogeschool van Amsterdam is mijn keuze gevallen op het uitvoeren van een project bij het bedrijf waar ik stage had gelopen, namelijk Fullmoon Interactive Solutions te Amsterdam.

Tijdens mijn half jaar durende stageperiode merkte ik af en toe dat men bij Fullmoon regelmatig tegen wat problemen aanliep, wat vooral veroorzaakt wordt door het feit dat er vrij kort op elkaar meerdere mensen waren aangenomen. Met acht vaste werknemers en twee stagiaires verliep het samenwerken daarom niet altijd even goed. Eén van de oorzaken was dat een goed overzicht van de verdeling van de werkzaamheden over de verschillende werknemers ontbrak. Bovendien kwam het regelmatig voor dat, wanneer men aan hetzelfde project bezig was, men per ongeluk werk van anderen overschreef, zodat er veel tijd en inspanning verloren ging.

Omdat ik uit eigen interesse al ervaring had opgedaan met Source Control Management  ({\sc scm}) systemen en veel aandacht besteedde aan mijn eigen workflow, heb ik Fullmoon aangeboden te helpen om hun situatie te kunnen verbeteren.

In mijn rapport getiteld ``Adviezen ter realisatie van een professionele technische infrastructuur voor webdevelopmentbureaus'' zijn de adviezen opgenomen, die zijn opgesteld aan de hand van mijn verrichtingen bij Fullmoon. Uitgangspunt bij dit rapport was om de veranderingen en verbeteringen die inmiddels zijn doorgevoerd bij Fullmoon duidelijk zichtbaar te maken, voor zowel studenten als voor bedrijven die met dezelfde problematiek te maken hebben.

In dit document wordt aan de hand van het adviesrapport in kaart gebracht wat er bij Fullmoon uiteindelijk geïmplementeerd is. Tot slot is er nog een persoonlijk projectverslag, waarin na te lezen is wat ik daadwerkelijk gedaan heb voor dit project.
