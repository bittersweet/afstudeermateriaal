\section{Inleiding}

Om af te studeren aan de opleiding Interactieve Media had de keuze gemaakt om een project uit te voeren bij het bedrijf waar ik stage had gelopen, Fullmoon Interactive Solutions te Amsterdam. Tijdens mijn half jaar durende stage merkte ik af en toe dat Fullmoon tegen wat problemen aanliep omdat ze recent meer mensen hadden aangenomen. Met acht vaste werknemers en twee stagiaires verliep het samenwerken niet altijd even goed. Dit lag voornamelijk aan het feit dat er geen goed overzicht was wie waar aan bezig was en dat mensen per ongeluk werk van anderen overschreven als ze aan hetzelfde project bezig waren.

Omdat ik uit eigen interesse al ervaring had met Source Control Management systemen, en veel aandacht besteedde aan mijn eigen workflow, had ik voorgesteld om Fullmoon te helpen met hun groeiproblemen.

In mijn adviesrapport getiteld ``Adviezen ter realisatie van een professionele technische infrastructuur voor webdevelopment bureaus'' zijn adviezen na te lezen die opgesteld zijn aan de hand van de veranderde situatie bij Fullmoon. Dit rapport had als uitgangspunt om het verrichte werk bij Fullmoon ook voor andere bedrijven en studenten van belang te kunnen laten zijn. In dit document wordt aan de hand van dit adviesrapport in kaart gebracht wat er bij Fullmoon uiteindelijk geïmplementeerd is. Hierna volgt een persoonlijk projectverslag.

\section{Implementatie}

In het adviesrapport zijn een vijftal adviezen opgesteld, wat volgt is uitleg over hoe deze zijn uitgewerkt bij Fullmoon.

\subsection{SCM Systeem}

keuze gevallen op SVN
koppeling met Redmine en Capistrano
automatische update naar testomgeving

\subsection{Ontwikkelstraat}

4 gescheiden omgevingen, maakt gebruik van SVN om bepaalde omgevingen te draaien.

\subsection{Backup}



\subsection{Ticketsysteem}

Redmine waar een overzicht te zien is van alle projecten en waar activiteit bekeken kan worden, tickets etc.

\subsection{Automatisch deployen}

Capistrano dat gebruikt maakt van SVN en code naar alle omgevingen kan deployen.
