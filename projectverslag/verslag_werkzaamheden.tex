\chapter{Verslag werkzaamheden}

In dit twee deel van het projectverslag zal ik kort in kaart brengen wat ik allemaal voor m'n afstudeerproject heb gedaan. Naast een veranderde situatie hoort hier natuurlijk ook het werk voor het adviesrapport bij. 

Omdat ik al een half jaar stage had gelopen bij Fullmoon had ik al een redelijk idee van de wensen en eisen maar ik heb erg vaak vergaderd samen met Bart Brugmans om deze goed in kaart te kunnen brengen. Vaak had ik een aantal alternatieven voor een bepaald probleem onderzocht waarna ik samen met Bart tot de beste keuze kon komen.

Om de voordelen van wat ik in wilde voeren duidelijk te maken aan de werknemers van Fullmoon heb ik nog tijdens mijn stage een document opgeleverd waarin ik schreef over de huidige situatie en hoe deze verbeterd kon worden door middel van {\sc scm}. Hierna heb ik ook nog een presentatie gegeven over hoe dit in zijn werk zal gaan en vragen beantwoord.

Nadat er samen met Fullmoon beslist was over de verschillende dingen die ingevoerd gingen worden, onder andere het gescheiden werken en gebruik maken van Subversion, kon ik over gaan tot het daadwerkelijke inrichten van de situatie. Zo heb ik een server volledig klaar gemaakt om de Subversion repository en de Redmine site te hosten. Redmine heb ik geconfigureerd naar de wensen van Fullmoon, zo is er een koppeling met Subversion en maakt het gebruik van de mailfuncties van de server om updates van tickets en dergelijke naar medewerkers te sturen. Om de workstations heb ik overal een webserver geïnstalleerd (Xampp) en heb ik deze klaar gemaakt voor het gebruiken van Subversion.

Ook heb ik een tiental lopende projecten in Subversion geïmporteerd en deze zo aangepast dat ze ook van de lokale webserver te draaien zijn, normaal gezien werkte deze alleen maar op één centrale server. Om de meest gebruikte handelingen zoals het aanmaken van een nieuw project, importeren van data en het echte werken met Subversion nog meer te verduidelijken heb ik een interne wiki gevuld met informatie. Hier kunnen de medewerkers van alles nog even nalezen mochten ze er niet meer uitkomen.

Omdat er samen met Fullmoon beslist was dat de situatie nu klaar was voor gebruik heb ik in samenwerking met Bart nog een presentatie gegeven. Naast Subversion ging dit over de veranderingen ten opzichte van de oude situatie doordat er met een lokale webserver gewerkt ging worden, en het gebruik van Redmine.

Naast informatie op de wiki heb ik ook alle medewerkers een half tot een heel uur individueel begeleid. Zo konden ze beter in de gaten krijgen hoe ze met Subversion moesten werken, want naar een presentatie kijken is toch minder duidelijk dan dat je er zelf mee aan de slag gaat. Ook heb ik hier nog heel wat vragen beantwoord, waarna ik vragen die meer voorkwamen nog extra uitgewerkt heb op de wiki.