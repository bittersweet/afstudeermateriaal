\chapter{Werkzaamheden}

In dit twee deel van het projectverslag zal ik kort in kaart brengen wat ik allemaal voor m'n afstudeerproject heb gedaan en wat ik hier van opgestoken heb.

\section{Verrichtingen}

Na een half jaar stage had ik al een redelijk idee van de wensen en eisen maar om hier zeker van te zijn heb ik erg vaak vergaderd met Bart Brugmans. Vaak had ik een aantal alternatieven voor een bepaald probleem onderzocht waarna ik samen met Bart tot de beste keuze kon komen.

Om de voordelen van wat ik wilde invoeren duidelijk te maken aan de werknemers van Fullmoon heb ik nog tijdens mijn stage een document opgeleverd waarin ik schreef over de huidige situatie en hoe deze verbeterd kon worden door middel van {\sc scm}. Hierna heb ik ook nog een presentatie gegeven over hoe dit in zijn werk zal gaan en heb ik vragen beantwoord.

Nadat er samen met Fullmoon beslist was over de verschillende zaken die ingevoerd moesten worden, onder andere het gescheiden werken en gebruik maken van Subversion, kon ik over gaan tot het daadwerkelijke inrichten van de situatie. Zo heb ik een server volledig klaar gemaakt om de Subversion repository en de Redmine site te hosten. Redmine heb ik geconfigureerd naar de wensen van Fullmoon. Zo is er een koppeling met Subversion en maakt het gebruik van de mailfuncties van de server om updates van tickets en dergelijke naar medewerkers te sturen. Op de workstations heb ik overal een webserver geïnstalleerd en heb ik deze klaar gemaakt voor het gebruik van Subversion.

Ook heb ik een tiental lopende projecten in Subversion geïmporteerd en deze zo aangepast dat ze ook van de lokale webserver te draaien zijn, normaal gezien werkte deze alleen maar vanaf de centrale server. Om de meest gebruikte handelingen zoals het aanmaken van een nieuw project, importeren van data en het echte werken met Subversion nog meer te verduidelijken heb ik een interne wiki gevuld met informatie. Hier kunnen medewerkers rustig de belangrijkste dingen nalezen mochten ze er niet meer uitkomen.

Omdat er samen met Fullmoon beslist was dat de situatie op dit moment klaar was voor gebruik heb ik in samenwerking met Bart nog een presentatie gegeven. Naast Subversion ging dit over de veranderingen ten opzichte van de oude situatie doordat er met een lokale webserver gewerkt ging worden en het gebruik van Redmine.

Naast informatie op de wiki heb ik ook alle medewerkers een half, of soms een uur lang, individueel begeleid. Zo konden ze beter in de gaten krijgen hoe ze met Subversion moesten werken, want naar een presentatie kijken is toch minder duidelijk dan dat je er zelf mee aan de slag gaat. Ook heb ik hier nog heel wat vragen beantwoord, waarna ik vragen die vaker voorkwamen nog extra uitgewerkt heb op de wiki.

Na een testperiode heb ik feedback van werknemers verzameld om te peilen hoe het verliep. Het meeste was positief maar er waren een paar onderwerpen die sommige het liefst nog iets duidelijker uitgelegd wilden krijgen. Het ging hierover zaken als wat te doen bij een conflict (in Subversion) en hoe het beste te branchen. Deze vragen en onderwerpen waarvan ik dacht dat ze misschien ook nog moeite mee hadden heb ik uitgewerkt en extra belicht op de wikipagina.

\section{Opgestoken kennis}

Naast dat ik nu vrij veel over het inrichten van een technische infrastructuur af weet, zijn er nog een aantal dingen waarover ik blij ben als ik nu reflecteer over het project. In principe heb ik het hele vierde jaar goed zelfstandig volbracht. Ik heb een afstudeerproject voorgesteld bij een bedrijf, hun situatie in kaart gebracht, adviezen voorgesteld om dit te verbeteren en een hele planning. Vooral over dat laatste ben ik blij, want de planning bleek een goede te zijn, waar ik me volledig aan heb kunnen houden. In de voorgaande studiejaren had ik niet altijd de discipline om opgestelde planningen te volgen, hierdoor was het soms nachtwerk voor een deadline. Dingen die af moesten komen werden soms uitgesteld. Natuurlijk deed ik een project bij een bedrijf, waardoor er wel wat van je verwacht wordt en er een zekere druk op de ketel staat, maar toch ben ik erg tevreden over mijn discipline en zelfstandigheid.

In het vierde en laatste jaar, krijg ik dus het gevoel dat ik een volwassen student ben geworden als het ware. Wat dat betreft komt het wel goed overeen met het leerpatroon dat de studie heeft opgesteld.

Naast het houden aan een planning komt deze volwassenheid ook in andere zaken terug. Zo heb ik nadat ik de situatie bij Fullmoon in kaart had gebracht, mijn visie over de verbetering hiervan gegeven. In voorgaande jaren had ik nooit een bedrijf kunnen adviseren over hun werksituatie, toen voelde ik me nog teveel student, waarbij ik me nu een volwaardige collega voel. 

Dit merkte ik ook bij het omgaan met andere medewerkers. Ik heb individueel en in groepen trainingen en uitleg gegeven. Hierbij was ik niet bang om mijn mening te laten horen. Er wordt wel veel met projecten gewerkt op de opleiding, toch is het omgaan met studenten niet hetzelfde als met collega's. Ik ben van mening dat ik hierover veel heb opgestoken tijdens mijn stage en afstudeerproject.
