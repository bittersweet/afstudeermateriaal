\chapter{Werkzaamheden}

In dit tweede deel van het projectverslag zal ik kort in kaart brengen wat ik voor mijn afstudeerproject heb gedaan en wat ik hier van opgestoken heb.

\section{Verrichtingen}

Na een half jaar stage had ik al een aardig inzicht gekregen naar de wensen en eisen met betrekking tot de verbetering van de werksituatie. Ik heb getracht om aan de wens van Fullmoon om te komen tot een efficiëntere werkwijze, tegemoet te komen door na te denken over verbeteringen en aanpassingen. Om mijn ideeën hierover te kunnen toetsen heb ik veel kunnen brainstormen met Bart Brugmans, projectmanager bij Fullmoon. Vaak had ik verschillende alternatieven voor een bepaald probleem onderzocht, waarna Bart en ik samen overlegden om tot de beste keuze te komen. Om de voordelen van de door mij voorgestelde aanpassingen duidelijk te maken aan de medewerkers van Fullmoon, heb ik nog tijdens mijn stageperiode een document opgesteld, waarin ik de huidige situatie heb besproken en hoe deze verbeterd kon worden door middel van {\sc scm}. Daarnaast heb ik ook nog een presentatie gegeven over hoe dit in zijn werk zal gaan en heb ik vragen beantwoord. 

Tijdens het project kon ik, nadat in nauw overleg met Fullmoon beslist was over de diverse zaken die moesten worden ingevoerd, zoals het gescheiden werken en gebruik maken van Subversion, overgaan tot het daadwerkelijk inrichten van de situatie.

Zo heb ik een server volledig klaar gemaakt om de Subversion repository en de Redmine site te hosten. Redmine heb ik geconfigureerd naar de wensen van Fullmoon. Zo is er een koppeling met Subversion en maakt het gebruik van de mailfuncties van de server om updates van tickets en dergelijke naar medewerkers te sturen. Op de workstations heb ik overal een webserver geïnstalleerd en heb ik deze klaar gemaakt voor het gebruik van Subversion. Ook heb ik een tiental lopende projecten in Subversion geïmporteerd en deze zo aangepast dat ze ook van de lokale webserver te draaien zijn. Normaal gezien werkte deze alleen maar vanaf de centrale server. Deze projecten heb ik ook zo ingericht dat er gebruik gemaakt kon worden van Capistrano om deze gestroomlijnd naar verschillende omgevingen te kunnen uploaden.

Om de meest gebruikte handelingen zoals het aanmaken van een nieuw project, het importeren van data en het echte werken met Subversion nog meer te verduidelijken heb ik een interne wiki gevuld met informatie. Hier kunnen medewerkers rustig de belangrijkste dingen nalezen mochten ze er niet meer uitkomen.

In overleg met Fullmoon werd beslist dat de situatie klaar was voor gebruik, waarna ik in samenwerking met Bart nog een presentatie heb gegeven. Behalve over Subversion ging deze presentatie over de veranderingen ten opzichte van de oude situatie met betrekking tot de werkwijze en over het gebruik van Redmine.

Naast de uitleg tijdens de presentatie en de informatie op de wikipagina, heb ik ook alle medewerkers een half uur tot een uur individueel begeleid. Hierdoor kregen ze beter in de gaten hoe er in de praktijk met Subversion gewerkt moest worden, want naar een presentatie kijken is toch minder duidelijk dan dat je er zelf mee aan de slag gaat. Ook tijdens deze sessies heb ik nog heel wat vragen beantwoord, waarna ik vragen die vaker gesteld werden nog extra uitgewerkt heb op de wiki. Na een testperiode van enkele weken heb ik de feedback van de medewerkers verzameld om te peilen hoe het een en ander verliep. De meeste reacties waren positief, maar er waren een paar onderdelen waar men nog iets duidelijker uitleg over wilde krijgen, zoals wat te doen bij een conflict in Subversion en hoe men het beste kan branchen. Deze vragen en enkele onderwerpen waarvan ik vermoedde dat men er misschien nog moeite mee had, heb ik verder uitgewerkt en extra belicht op de wikipagina.

\section{Opgestoken kennis}

Afgezien van het feit dat ik nu vrij veel weet op het gebied van het inrichten van een technische infrastructuur, zijn er nog een aantal zaken waar ik tevreden over ben als ik terugkijk naar het project. In principe heb ik het gehele vierde studiejaar goed zelfstandig volbracht. Ik heb een afstudeerproject voorgesteld bij een bedrijf, hun werksituatie in kaart gebracht, adviezen aangedragen om bepaalde zaken te verbeteren en tenslotte een planning opgesteld. Vooral over die planning was ik tevreden, want het bleek een goede te zijn waar ik mij volledig aan heb kunnen houden. In de voorgaande studiejaren was niet altijd de discipline aanwezig om opgestelde planningen te volgen, waardoor het soms nachtwerk werd om een deadline te halen. Ik denk niet dat ik een uitzondering vormde op dit gebied in vergelijking met mijn medestudenten. Natuurlijk heb ik een project uitgevoerd bij een zakelijke instelling, waardoor er wel wat van je verwacht wordt en er een zekere druk op de ketel staat, maar toch ben ik erg tevreden over mijn ontwikkeling op het gebied van discipline en zelfstandigheid.

In het vierde en laatste studiejaar, heb ik het gevoel gekregen dat ik als student meer volwassen ben geworden. Wat dat betreft komt het wel goed overeen met het leerpatroon dat voor de studie is opgesteld.

Naast het vasthouden aan een planning komt deze groei naar volwassenheid mij ook ten goede in andere situaties. Zo heb ik nadat ik de werksituatie bij Fullmoon in kaart had gebracht, mijn visie over de verbetering kunnen geven. In voorgaande jaren had ik er nooit aan gedacht om een bedrijf te kunnen adviseren over hun werksituatie, omdat ik me nog teveel student voelde, terwijl ik me nu een volwaardige collega voel. 

Dit merkte ik ook bij de manier van omgaan met andere medewerkers. Ik heb individueel en in groepen met meer zelfvertrouwen trainingen en uitleg gegeven, zonder aarzeling om mijn mening te laten horen. Tijdens de opleiding wordt wel veel met projecten gewerkt, maar er bestaat toch verschil in het omgaan met studiegenoten en met collega's. Dat ik tijdens mijn stageperiode en afstudeerproject hier veel over heb opgestoken is voor mijn toekomstige carrière erg belangrijk.
