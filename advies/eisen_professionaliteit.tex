\chapter{Eisen professionaliteit}

\section{Introductie}

Voordat ik adviezen kan geven voor een goede werksituatie zal ik eerst in kaart brengen wat mijn eisen waren bij Fullmoon om goed professioneel te kunnen werken. De situatie hier is dus gericht op het midden- en kleinbedrijf (MKB), maar is in mindere mate ook zeker toepasbaar voor kleine groepen maar ook voor mensen die zelfstandig aan een project werken.

Hierna zal ik nog een lijst geven van bijkomende zaken die niet direct vereist zijn maar wel erg handig kunnen blijken voor een professionele manier van werken.

\section{Eisen}

\begin{enumerate}
  \item Medewerkers moeten gelijktijdig kunnen samenwerken aan hetzelfde project zonder elkaar hierbij in de weg te zitten.
  \item Er moet een overzicht zijn van wat er gebeurt: wie heeft wat aangepast en wanneer, waar is iedereen mee bezig?
  \item Er moet een goede ontwikkelsituatie zijn -- gescheiden in verschillende omgevingen -- gebaseerd op een OTAP \footnote{Ontwikkel, Test, Acceptatie, Productie} straat.
  \item Van alle werkzaamheden moeten goede backups gemaakt kunnen worden. Dit van zowel de productie- als ontwikkelomgevingen. Er moet geen versie van een bestand verloren kunnen gaan en men moet altijd terug kunnen keren naar een vorige versie van een bestand.
  \item Klanten moeten in het ontwikkelproces betrokken kunnen worden, zodat nieuwe projecten en aanpassingen beoordeeld kunnen worden in een omgeving die los staat van de productieomgeving.
\end{enumerate}

\section{Handige zaken}

\begin{itemize}
  \item Een ticketsysteem dat intern en extern bereikbaar is, zodat werknemers (en klanten) op een centrale plaats kunnen noteren wat er mis is aan een bepaald project en wat er nog moet gebeuren.
  \item Een eenvoudige manier om projecten door de verschillende omgevingen heen te loodsen zodat bestanden niet allemaal met de hand geupload hoeven te worden.
\end{itemize}
