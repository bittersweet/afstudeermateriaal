\chapter{Advies professionaliteit}

\section{Introductie}

Om aan de opgestelde eisen van professionaliteit te kunnen voldoen wil ik een aantal adviezen voordragen. Deze zal ik kort benoemen waarna ik er in hoofdstuk 4 dieper op in ga.

\section{Adviezen}

\begin{enumerate}
  \item Gebruik maken van een {\sc scm} systeem. Dit faciliteert gelijktijdig het beter kunnen samenwerken en zorgt er bovendien voor dat er een historie aanwezig is van veranderingen aan bestanden. Dit is ook een onmisbaar onderdeel om een goed overzicht te krijgen van wat iedereen doet.
  \item Medewerkers laten ontwikkelen op hun eigen workstations in plaats van op een centrale server. In samenwerking met het {\sc scm} systeem zal iedereen onafhankelijk van elkaar kunnen werken en toch wijzigingen van anderen kunnen binnenhalen.
  \item Een ontwikkelstraat gebruiken zodat een project meerdere omgevingen doorloopt. Naast de voordelen die dit biedt voor samenwerking en kwaliteitscontrole kan ook de klant in het ontwikkelproces betrokken worden door bijvoorbeeld wijzigingen uit te testen alvorens deze online komen.
  \item Er moet een goede mogelijkheid zijn om backups te maken zodat er geen informatie verloren kan gaan. Dit geldt voor zowel normale bestanden als informatie uit een database.
  \item Er moet een centraal support/ticket systeem komen waar zowel medewerkers als klanten toegang toe hebben. Hier kunnen bijvoorbeeld bugs en taken op aangemeld worden, die vervolgens toegewezen worden aan medewerkers. Zo heeft men een goed overzicht van taken en wordt er een transparant proces voor de klant geboden.
  \item Het updaten (``deployen'') van sites moet eenvoudig verlopen om er voor te zorgen dat het proces medewerkers niet in de weg zit en zodat fouten die kunnen ontstaan tijdens het handmatig updaten vermeden kunnen worden.
  
\end{enumerate}
