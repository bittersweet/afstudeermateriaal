\chapter{Advies professionaliteit}

\section{Introductie}

Om aan de opgestelde eisen van professionaliteit te kunnen voldoen wil ik een aantal adviezen voordragen. Deze zal ik kort uitleggen waarna ik er in hoofdstuk vier dieper op in ga.

\section{Adviezen}

\begin{enumerate}
  \item Gebruik maken van een {\sc scm} systeem. Dit faciliteert gelijktijdig het beter kunnen samenwerken en zorgt ervoor dat er een historie aanwezig is van veranderingen en bestanden. Om een goed overzicht te krijgen van wat iedereen doet is dit ook een onmisbaar onderdeel.
  \item Medewerkers laten ontwikkelen op hun eigen workstations in plaats van op een centrale server. In samenwerking met het {\sc scm} systeem zal iedereen onafhankelijk van elkaar kunnen werken en toch wijzigingen van anderen kunnen binnenhalen.
  \item Een ontwikkelstraat gebruiken zodat een project meerdere omgevingen door gaat. Naast de voordelen die dit biedt voor samenwerking en kwaliteitscontrole kan ook de klant in dit proces betrokken worden door bijvoorbeeld wijzigingen uit te testen alvorens deze online komen.
  \item Er moet een goede backupmogelijkheid zijn zodat er geen informatie verloren kan gaan. Dit voor zowel normale bestanden als informatie uit een database.
  \item Er moet een centraal support/ticket systeem komen waarbij zowel medewerkers als klanten toegang toe hebben. Hier kunnen bijvoorbeeld bugs en taken op aangemeld worden die vervolgens toegewezen worden aan medewerkers. Zo heb je een goed overzicht van taken en bied je een transparant proces voor je klant.
  \item Het updaten ('deployen') van sites moet geautomatiseerd worden, om fouten die kunnen ontstaan tijdens het handmatige updaten uit de weg te gaan.
  
\end{enumerate}
