\chapter{Conclusie}

Met dit document heb ik de succesvolle workflow verandering bij Fullmoon om willen zetten in een document dat voor meerdere mensen bruikbaar is. De werknemers bij Fullmoon kunnen inmiddels zonder problemen samenwerken aan hetzelfde project en er is een centrale plek waar ze een goed overzicht kunnen krijgen van werkzaamheden en taken.

Ik heb mijn best gedaan om een aantal goede adviezen op te stellen en ik ben van mening dat deze echt iets kunnen toevoegen aan de verdere professionalisering van bedrijven en studenten. Met goede mogelijkheden tot samenwerking door verschillende ontwikkelomgevingen en {\sc scm}, een overzicht van werkzaamheden en taken, de optie om klanten in het ontwikkelproces te betrekken en de zekerheid van goede backups wordt een goede bedrijfsvoering mogelijk.

Als er nog niet met een {\sc scm} systeem gewerkt wordt, zal dat de grootste en ook meest merkbare verandering zijn. Uit eigen ervaring kan ik zeggen dat het invoeren hiervan echt een nieuwe impuls geeft aan het plezier van werken. Het geeft je een goed gevoel om te kunnen zien wat je precies gedaan hebt en het geeft je zekerheid. Sinds ik gebruik maak van source control werk ik zonder zorgen aan aan websites, met in mijn achterhoofd dat ik geen fouten kan maken aangezien ik altijd een stabiele versie heb. Bij Fullmoon heerst dit gevoel ook.

Ook al draait een bedrijf op dit moment goed, er moet toch naar de toekomst gekeken worden. Stel dat er in het komende jaar drie mensen worden aangenomen, is de huidige manier van werken dan nog wel toereikend? Fullmoon heeft er goed aan gedaan om de situatie nu aan te passen, en is klaar voor de toekomst en om verder uit te groeien.

Studenten heb ik hopelijk met dit document nieuwe ideeën en inspiratie gegeven om ook naar hun eigen manier van werken te kijken om te zien of en hoe ze deze kunnen verbeteren.