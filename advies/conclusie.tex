\chapter{Conclusie}

Met dit document heb ik de succesvolle workflow verandering bij Fullmoon om willen zetten in een document dat voor meerdere mensen bruikbaar is. Ik heb mijn best gedaan om er een aantal goede adviezen in te zetten en ik ben van mening dat dit document echt iets kan toevoegen aan de professionalisering van bedrijven en studenten.

Als er nog niet met een {\sc scm} systeem gewerkt wordt, zal dat de grootste verandering zijn, en ook de meest merkbare. Uit eigen ervaring kan ik zeggen dat het invoeren hiervan echt een nieuwe impuls door het plezier van werken te vergroten. Het geeft een goed gevoel als je een precieze historie in kan kijken van het werk wat je verricht en het geeft je een zeker gevoel. Sinds ik gebruik maak van source control werk ik zonder zorgen aan verbeteringen aan websites, met in mijn achterhoofd dat ik het toch niet kan verpesten aangezien ik altijd nog een stabiele versie heb. Bij Fullmoon heerst dit gevoel ook.

Ook al draait een bedrijf op het moment goed, er moet naar de toekomst gekeken worden. Stel dat er in het komende jaar drie mensen worden aangenomen, is de huidige manier van werken dan nog wel toereikend? Fullmoon heeft er goed aan gedaan om de situatie nu aan te passen, en ze zijn klaar voor de toekomst en om verder te groeien.

Als je goede samenwerkingsmogelijkheden hebt, een overzicht van werkzaamheden, verschillende ontwikkelomgevingen, zekerheid over je data en de mogelijkheid om klanten in het ontwikkelproces te betrekken, heb je een professioneel bedrijf!