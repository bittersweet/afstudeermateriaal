\chapter{Uitgewerkte adviezen}

\section{Introductie}

In dit gedeelte van het document zal ik dieper in gaan op de adviezen die ik net heb opgesteld. Ik zal een aantal opties aandragen waarna ik mijn persoonlijke voorkeur uitspreek.

Er is geen \'{e}\'{e}n bedrijf hetzelfde, iedereen werkt bijvoorbeeld weer met andere programmeertalen. Om de grootte van dit document niet ernstig uit de klauwen te laten lopen zal ik mij richten op twee werkomgevingen waarbij een Interactieve Media student terecht kan komen. Namelijk, een bedrijf wat developed in PHP en eentje die werkt met Ruby on Rails.

\section{{\sc scm} systeem}

Gebruik maken van een {\sc scm} systeem bied ontzettend veel voordelen. Het opslaan van verschillende versies van bestanden tijdens het ontwikkelproces is van onmisbaar belang. Het biedt je veiligheid want je kan altijd terug gaan naar een vorige versie als er bijvoorbeeld een fout in de code is geslopen. Het zorgt er ook voor dat je goed kan samenwerken omdat elke werknemer individueel kan werken aan zijn of haar eigen versie van een project. Hierdoor kan het ook een overzicht bieden van wie wat doet, je kan precies zien wat is aangepast in een bepaald bestand en door wie.

Naast deze op zich voor de hand liggende voordelen biedt het ook stabiliteit. Doordat je met een {\sc scm} systeem de mogelijkheid hebt om te \emph{branchen} kan je gelijktijdig meerdere versies van een codebase hebben. Bijvoorbeeld \emph{stable} en \emph{development}. Doordat je wijzigingen in een speciale werk branch uitvoert, heb je altijd een stabiele branch. Nadat alle wijzigingen uitvoerig zijn getest kunnen ze in de stabiele versie worden ge\"{i}ntegreerd. Dit bevorderd ook experimentatie, omdat het weinig moeite kost om te schakelen tussen verschillende versies en omdat het veilig is, is het ``goedkoop'' om dingen uit te proberen.

\section{Gescheiden werken}

\section{Ontwikkelstraat}

\section{Backup}

\section{Ticketsysteem}

\section{Automatisch deployen}